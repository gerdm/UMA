\documentclass{article}
\usepackage[utf8]{amsmath}
\usepackage{mathtools}

\title{Notes on Pricing Derivatives}
\author{Gerardo Durán Martín}

\newtheorem{theorem}{Theorem}[section]

\begin{document}
\maketitle


\section{Introduction}
Broadly speaking, financial instruments can be grouped into two distinct categories: \textit{underlying}s and \textit{derivatives}. The former referfs to stocks, bonds, commodities, foreign exchange currencies, etc; while the latter refers to financial contracts with a promise to pay in the future, whose value is \texit{derived} from some underlying asset. This last point imples, that the aformentioned future payment is contingent to the random movement of the underlying, whichever this might be.\\

As a first attempt to \texit{price} any of these derivatives, one would consider \textbf{Kolmogorov's Strong Law of Large Numbers}:

\begin{theorem}
    Let $\{X_n\}_{n\geq 1}$ be a collection of i.i.d. random variables with mean $\mu$. Denote $S_n = \frac{1}{n}\sum_{i=1}^n X_n$. Then, with probability one,
    \begin{equation}
        S_n \xrightarrow[n \rightarrow \infty]{}\mu
    \end{equation}
\end{theorem}

Since the future payoff is random and, assuming that each payoff has the same distribution, pricing this contingent future payoff as the expected value 
\end{document}
